\chapter{Discussion}
\label{chapter:discussion}

This report had the goal of explaining the implementation of AltiVec intrinsic functions. It started straight forward with explaining the basic structure of compilers such as GCC and emphasized on the rs/6000 back-end. This followed by a step-by-step guide on how intrinsic functions are build bottom-up and added the case of special builtins.
The value of intrinsic functions can be questioned when compared to inline functions and inline assembly but in any way intrinsics do allow calling specific actions on the processor that are otherwise not available while also allowing for optimization by the compiler. This generates the perfect use case for extensions such as AltiVec where different versions of add instruction are crucial to the usefulness of such an extension. Intrinsics therefore provide the common user to take control over the Processing Unit in a simple way that neither requires knowledge of assembly nor forces the user to produce efficient code by himself.

This is contrary to the current state of PPU support which demands assigning registers by hand as well as choosing the right memory address. In comparison this is more complicated and prone to inefficiency when done by an unexperienced user especially for large programs.

As it is obvious that this report does not help a lot in understanding how a compiler back-end completely works but rather is a tutorial on how to add intrinsic functions to the rs/6000 back-end it is meant to rather advertise the future use of implementation of intrinsic functions for the PPU. Though it also did not show all possibilities there are for intrinsics, most of these are barely needed and it would require further knowledge of a back-end. This goes beyond what is needed in case of the reduced instruction set of the PPU.
Ultimately this guide may help the most when there is an instruction set to be added to the rs/6000 back-end and to give an overview of what may be possible to include in a GCC back-end.
We leave further explanation of the internals of GCC to the official GCC internals handbook \cite{GCCinternals} as it provides a more detailed look into RTL and insn defintion.
