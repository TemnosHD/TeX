\documentclass[10pt,aspectratio=169]{beamer}

\usepackage{presento/config/presento}
\usepackage{tikz}
\usetikzlibrary{shadings,shadows,calc,backgrounds,positioning}
\usepackage{tcolorbox}
\usepackage{adjustbox}
\tcbuselibrary{skins,breakable}
\usepackage[small, format=hang,justification=raggedright]{caption}

\usepackage[caption=false]{subfig}
%\usepackage{transparent}
\setbeameroption{show notes}

\usepackage[backend=bibtex]{biblatex}
\bibliography{bib/myrefs}

% figures
\usepackage {pgf}                                 % includepgf for bitmaps
\usepackage{graphicx}

% Information
\title{Extending Compiler Support for the BrainScaleS Plasticity Processor}
\subtitle{Bachelor's Thesis Presentation}
\author{Arthur Heimbrecht}
\institute{}
\date{\today}

\begin{document}


% Title page
{
\usebackgroundtemplate{
    \tikz[overlay, remember picture]\node[opacity=0.3, inner sep=0, outer sep=0, ] at (current page.center) {\includegraphics[width=\paperwidth]{pictures/dls_die_photo.jpg}};
}
\begin{frame}[plain]
\maketitle
\note{ Welcome everybody to this talk about my Bachelor thesis. As this is a quite technical talk, feel free to ask questions at any time.}
\note {The topic of my thesis was Extending Compiler Support for the BrainScaleS Plasticity Processor.}
\end{frame}

}

\begin{frame}{Contents}
\begin{columns}[c]
    \begin{column}{.5\textwidth}
        \begin{minipage}[t][0.5\textheight]{0.75\textwidth}
            \tableofcontents[sectionstyle=show, subsectionstyle=show/show/shaded]
        \end{minipage}\hfill
    \end{column}

    \begin{column}{.5\textwidth}
        \centering
        \begin{figure}
            \includegraphics[width=\textwidth]{pictures/dls_die_photo.jpg}
            \caption{\label{fig:dls} Photograph of HICANN-DLS chip, \citeauthor{PPU}}
        \end{figure}
    \end{column}
\end{columns}
\note{
    as the title hints there are two main components to this talk, which are the plasticity processing unit (PPU) and Compilers.
    I will briefly talk about both of these and their applications.
    Afterwards I will explain, what I did during my thesis, which is of course followed by a short presentation of the result.
}
\end{frame}

% sections in the presentation
\section{PPU Architecture}
\begin{frame}{PPU Architecture}
    \begin{columns}[c]
    \begin{column}{0.5\textwidth}
        \begin{itemize}
            \item The design is \underline{clean}
        \end{itemize}
    \end{column}

    \begin{column}{0.5\textwidth}
        \centering
        \vspace*{3em}
        \begin{figure}
                \begin{adjustbox}{center, max width={.7\columnwidth}}
                    \chapter{Processor basics}
\label{chapter:processor}

Next to all processors used these days are built upon the so called von-Neumann architecture \todo{add reference}.
\todo{cite freidmann dissertation}Though the main goal of this group is to provide an alternative analogue architecture that is inspired by nature, there are advantages to the classic model of processors which are needed at some point.
The main advantage of digital systems over analogue systems such as the human brain, is the ability to do calculations at much higher speeds.
For this reason ``normal" processors are responsible for handling experiment data as well as setting up different parts of the experiment.
One such task is applying learning rules to the synapses during or in between experiments which can either be done by hand or with the help of the aforementioned PPU.
The second option is especially valuable when updating synaptic weights during an experiment as the PPU does this much faster than a system which interacts from the outside.
This is important for achieving experimental speeds that are $10^{4}$ times faster than their biological counterparts.

Therefore the PPU is one of many von-Neumann processors in this world and follows the same basic concepts.
It is important to understand these concepts as they build the foundation to this report!



ALU
FPU
memory
memory controller
clockcycle
pipeline


                \end{adjustbox}
            \caption{\label{fig:processor} Schematic of Processor in von-Neumann Architecture}
        \end{figure}
    \end{column}
    \end{columns}

\end{frame}

\begin{frame}{PPU Architecture}
    \begin{columns}[c]
    \begin{column}{0.5\textwidth}
        \begin{itemize}
            \item \begin{center}\largetext{The design is \underline{clean}}\end{center}
            \item \begin{center}\largetext{The rules are \underline{simple}}\end{center}
            \item \begin{center}\largetext{The code is \underline{extensible}}\end{center}
        \end{itemize}
    \end{column}

    \begin{column}{0.5\textwidth}
        \centering
        \begin{figure}
            \includegraphics[width=.8\textwidth]{pictures/nux.pdf}
            \caption{\label{fig:nux} Structure of nux Architecture}
        \end{figure}
    \end{column}
    \end{columns}

\end{frame}



\section{GCC Structure}
\begin{frame}[fragile]{GCC Structure}
    \begin{columns}[c]
    \begin{column}{0.5\textwidth}
        \begin{itemize}
            \item \begin{center}\largetext{The design is \underline{clean}}\end{center}
            \item \begin{center}\largetext{The rules are \underline{simple}}\end{center}
            \item \begin{center}\largetext{The code is \underline{extensible}}\end{center}
        \end{itemize}
    \end{column}

    \begin{column}{0.5\textwidth}
        \centering
        \begin{figure}
            \begin{adjustbox}{center, max width={.5\columnwidth}}
                    \tcbset
    {my box/.style={enhanced,colframe=blue!70!black,colback=white!50!blue,colupper=red!50!black,
        fonttitle=\bfseries,nobeforeafter,center title, noparskip, size=small},
    every box on layer 1/.style={every box},
    every box on layer 2/.style={reset,my box}}
\begin{tcolorbox}[enhanced jigsaw, width=\textwidth, opacityframe=0.0, opacityback=0.0]
\begin{tcolorbox}[enhanced, height=.5cm, width=\linewidth, remember as=pp, opacityframe=0.0, opacityback=0.0]\end{tcolorbox}
\begin{tcolorbox}[enhanced, height=.7cm, width=\linewidth, watermark text=Preprocessor, remember as=pp]\end{tcolorbox}
\begin{tcolorbox}[tcbox raise base, width=\linewidth, enhanced jigsaw, remember as=cmp]
    \begin{tcolorbox}[enhanced, breakable, noparskip,opacityframe=0.3, opacityback=0.3, height=1.4cm, width=\linewidth, watermark text=Front-End, remember as=fe]
    \end{tcolorbox}
    \begin{tcolorbox}[enhanced, breakable, noparskip,opacityframe=0.3, opacityback=0.3, height=0.7cm, width=\linewidth, watermark text=Middle-End, remember as=me]
    \end{tcolorbox}
    \begin{tcolorbox}[enhanced, breakable, noparskip,opacityframe=0.3, opacityback=0.3, height=1.1cm, width=\linewidth, watermark text=Back-End, remember as=be]
    \end{tcolorbox}
\end{tcolorbox}

\end{tcolorbox}

\begin{tikzpicture}[overlay,remember picture,line width=1mm]
    \draw[->, shorten >=-1.5mm] ($(pp.north)+(0,1.5)$) -- node [left] {program code} (pp.north);
    \draw[->, shorten >=-1.5mm] (pp.south) -- (fe.north);
    \draw[->, shorten >=-1.5mm] (fe.south) -- (me.north);
    \draw[->, shorten >=-1.5mm] (me.south) -- (be.north);
    \draw[->] (be.south) -- node [left] {machine files} ++(0,-1.5);
\end{tikzpicture}

                
    \tcbset
    {enhanced,colframe=blue!70!black,colback=white!50!blue,colupper=red!50!black,
    fonttitle=\bfseries,nobeforeafter,center title, noparskip}
            \begin{tcolorbox}[tcbox raise base, width=\linewidth, enhanced jigsaw, remember as=cp]
                \begin{tcolorbox}[enhanced jigsaw, breakable, noparskip,opacityframe=0.3, opacityback=0.3, width=\linewidth]
                    \begin{tcolorbox}[enhanced,center title,width=\linewidth, remember as=sc]
                        \begin{center}Scanner\end{center}
                    \end{tcolorbox}
                    \begin{tcolorbox}[enhanced,width=\linewidth, remember as=ps]
                        \begin{center}Parser\end{center}  
                    \end{tcolorbox}
                    \begin{tcolorbox}[enhanced, width=\linewidth, remember as=sa]
                        \begin{center}Semantic \\ Analyzer \end{center} 
                    \end{tcolorbox}
                    \begin{tcolorbox}[enhanced, width=\linewidth, remember as=sco]
                        \begin{center}Source Code \\ optimizer \end{center}
                    \end{tcolorbox}
                \end{tcolorbox}
                \begin{tcolorbox}[enhanced jigsaw, breakable, noparskip,opacityframe=0.3, opacityback=0.3, height=1cm, width=\linewidth, remember as=me, watermark text=Middle-End]
                \end{tcolorbox}
                \begin{tcolorbox}[enhanced jigsaw, breakable, noparskip,opacityframe=0.3, opacityback=0.3, width=\linewidth]
                    \begin{tcolorbox}[enhanced, width=\linewidth, remember as=cg]
                        \begin{center}Code Generator  \end{center}
                    \end{tcolorbox}
                    \begin{tcolorbox}[enhanced, width=\linewidth, remember as=tco]
                        \begin{center}Target Code \\ Optimizer \end{center}
                    \end{tcolorbox}
                \end{tcolorbox}
            \end{tcolorbox}
            \begin{tikzpicture}[overlay,remember picture,line width=1mm]
                \draw[->, shorten >=-1.5mm] ($(cp.north)+(0,1)$) -- (sc.north);
                \draw[->, shorten >=-1.5mm] (sc.south) -- (ps.north);
                \draw[->, shorten >=-1.5mm] (ps.south) -- (sa.north);
                \draw[->, shorten >=-1.5mm] (sa.south) -- (sco.north);
                \draw[-] (sco.south) -- (me.north);
                \draw[->, shorten >=-1.5mm] (me.south) -- (cg.north);
                \draw[->, shorten >=-1.5mm] (cg.south) -- (tco.north);
                \draw[->] (tco.south) -- ($(cp.south)+(0,-1)$);
            \end{tikzpicture}
            \begin{tikzpicture}[overlay,remember picture,line width=1mm]
                \draw[-, draw=blue!30!white,line width=.5mm, shorten >=.2cm,shorten <=.1cm] (cmp.north east) -- (cp.north west);
                \draw[-, draw=blue!30!white,line width=.5mm, shorten >=.2cm,shorten <=.1cm] (cmp.south east) -- (cp.south west);
            \end{tikzpicture}

            \end{adjustbox}
            \caption{\label{fig:compiler} Structure of Compiling Process and Compiler}
        \end{figure}
    \end{column}
    \end{columns}
\end{frame}



\section{Extending GCC for the PPU}
\begin{frame}{Extending GCC for the PPU}
 \begin{center}
 \end{center}
\end{frame}

\section{Results}
\begin{frame}{Results}
 \begin{center}
 \end{center}
\end{frame}



\section{References}
\begin{frame}{References}
    \printbibliography
\end{frame}

\end{document}
