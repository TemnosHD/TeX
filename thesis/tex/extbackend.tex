\chapter{Extending the GCC Back-End}
\label{chapter:extbackend}

To allow the compiler to differ which insns we acutally want to use as a built-in, the rs/6000 back-end has a special file which contians macros for every builtin function that will be available.
These  macros will be loadad in the rs6000.c file and also handled there.
But before this can happen we have to create a built-in macro for our newly created insn.
WE must first differ between the different kinds of macros.
The rs/6000 back-end already provides us with templates for differnt altivec built-ins.
These templates basically help identifying the right macros later on and avoid naming conflicts.
Next we have to differ between the number of input opernad our built-in function is going to have and the number  in th etemplate name is meant to be equal to th enumber of input operands .
all Macros expect output operand except for the X-names macros.
which are special cases we will emphasize later.
We now choose |BU_ALTIVEC_2| because our newest insn has to input operands and add the following lines:

the first argument is the name of the macro and will be used by the compiler internally.
the user only gets interact with the second string which is the new built-ins name for the front end.
(for altivec built-ins the tempalte adds |__builtin_altivec_| in front of the given name).
Next the attribute for the builtn must be given.
typically this |CONST| but for builtins that load or store in the memory this is |MEM| for example.
The last argument is the name of the insn we decided on earlier.

You can also add an „overload“ macro for your builtin as this allows for type checking and reducing builtins for different vector modes to just on final builtin.
here we first need a Macro name and second the function name we want to use later (this functon name is preceeded by |__builtin_vec_|)


After we fully implemented our builtin functions we will finalize the functions by defining their final names.
There are two naming conventions we can follow and typically it is advised to follow both in order to offer the user a set of alternatives.
The first naming convention is the nameing convention already used by the AltiVec extension which starts every builtin function with vec\_ and chooses a general descriptive name which is derived from the used Assembly macro if possible.
This is the most intresting case for our exemplary Altivec builtin function.
because there is already a vec\_ mul builltin we will choose the descriptive abbreviation vec\_ smul for scalar multiplication.
The other naming convention gets intresting when extending the backend with another vector extension.
If there were an S2PP vector extension, all assembly macros would begin with fxv which is a good alternative for the vec\_ prefix altivec uses for its builtins.
then it is a good idea so offer all assembly macros with their already known names e.g.
fxvaddbm would become fxv\_ addbm.
for more genral builtins that combine the halfword and byte macro it is recommended to leave out the letter h or b likewise.
rendering fxv\_ addm the builtin function that works with both V8HI and V16QI vectors.

important files:
\begin{description}
    \item[rs6000.md]
    \item[rs6000.h]
    \item[rs6000.c]
    \item[rs6000.opt]
    \item[rs6000-builtins.def]
    \item[rs6000-cpus.def]
    \item[rs6000-c.c]
    \item[rs6000-opts.h]
    \item[rs6000-protos.h]
    \item[rs6000-tables.opt]
    \item[driver-rs6000.c]
    \item[ppc-asm.h]
    \item[s2pp.md]
    \item[s2pp.h]
    \item[constraints.md]
    \item[predicates.md]
    \item[vector.md]
    \item[sysv4.h]
    \item[t-fprules] bla bla bla
\end{description}
        
