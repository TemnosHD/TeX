\chapter{Processor basics}
\label{chapter:processor}

Next to all processors used these days are built upon the so called von-Neumann architecture \todo{add reference}.
\todo{cite freidmann dissertation}Though the main goal of this group is to provide an alternative analogue architecture that is inspired by nature, there are advantages to the classic model of processors which are needed at some point.
The main advantage of digital systems over analogue systems such as the human brain, is the ability to do calculations at much higher speeds.
For this reason ``normal" processors are responsible for handling experiment data as well as setting up different parts of the experiment.
One such task is applying learning rules to the synapses during or in between experiments which can either be done by hand or with the help of the aforementioned PPU.
The second option is especially valuable when updating synaptic weights during an experiment as the PPU does this much faster than a system which interacts from the outside.
This is important for achieving experimental speeds that are $10^{4}$ times faster than their biological counterparts.

Therefore the PPU is one of many von-Neumann processors in this world and follows the same basic concepts.
It is important to understand these concepts as they build the foundation to this report!



ALU
FPU
memory
memory controller
clockcycle
pipeline

