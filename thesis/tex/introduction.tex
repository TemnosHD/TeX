\chapter{Introduction}
\label{chapter:introduction}

As neuromorphic computing becomes ever more popular, many applications for neuromorphic systems emerge and make use of neuromorphic hardware's advantages over traditional computers.
But to do so one must be familiar with a system as often it differs in many ways from common system architectures.
As a result programing for such systems can feel odd to new users as they need to abandon some familiar techniques and acquire new skills.
This understandably is a hurdle for many users and also initially takes a significant amount of time. 
The more a system's programming abandons common elements of programming which users have got used to the more this can become a problem.
Not only do fewer users take the initiative of writing for such systems but also can code easily get confusing, hard to debug and even ineffective.

An example of this is the current state of programming for the plasticity processing unit of the Hicann-DLS.
It is responsible for applying learning rules to neural systems on the Hicann-DLS and resembles a common processor which was extended for this cause.
As basic programming is still the same for this processor it differs for creating the mentioned learning rules.
Although these are still programmed in C a user needs to use a set of functions and predefined variables while at the heart of this is basic assembly programming.
This was done for the reason of not making a user learn assembly that wants to write programs for the PPU.
This lead to PPU programs having a distinct look and feel that is only in some regards similar to C but feels like being pushed back to the origins of computing; thing like reading out the value of a variable need unhandy workarounds and accessing memory is a repetitive set of program lines.
The main reason we feel such reluctant when it comes to ``antique'' programming are compilers.

What compilers have done and do for us every single day has become normal and we enjoy the state of not worrying about what a computer does at its core as long as we get a result at the end.
Hence the absence of a compiler strikes us quite hard.

But luckily the PPU is not without compiler support, though this does not apply completely.
The part which is responsible for computing learning rules -of all parts- is the one part that is not supported by any compiler there is.
This is due to the custom nature of the PPU, which was developed solely for the BrainScaleS project and the hicann-DLS.
A user therefore needs the most basic kind of programming to make use of the additional features of the PPU, which is the current state of programming.
In the worst case this can lead to ineffective programs - as performance is important for neuromorphic programming - and demand an unreasonable amount of time and work to achieve simple results.

The only way to fix this situation is adding support of the PPU to a compiler.

Not only could this allow for full C-style programming when working on the PPU but also include code optimization and code debugging.
We therefore aim to achieve ``full'' compiler support of the PPU's hardware and make programming easy again.

This thesis will focus on the way to achieve this and briefly explain the process itself.
As a fundamental knowledge of both processors and compilers is needed along this way we will start with a very basic introduction to both in general and also apply this to the specific processor and compiler we use.
This may not make actual literature for both obsolete but should explain the basic concepts to an extend which is sufficient for our cause.
The following chapter will then explain a few test scenarios that put the contests of the previous chapter to use and also make us more familiar with the PPU.
Afterwards the process of extending the compiler is explained and the result of this presented.
It follows a recap of what was done and an outlook to what might be done in the future.

