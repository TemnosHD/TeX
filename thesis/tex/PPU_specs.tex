\chapter{Insn-Coding}
\label{chapter:insn coding}

The PPU, which was designed by Simon Friedmann \todo{PPU paper}, is a custom processor, that is based on the Power Instruction Set Architecture (PowerISA), which was developed by IBM since 1990. The specifically the PPU uses POWER7 which is a successor of the original POWER architecture and was released in 2010.

The PPU's distinct feature is its vector unit or vector extension that allows for Single Input Multiple Data (SIMD) operations.
This was added due to the need for fast handling and writing of the synaptic weights into the array of synaptic values on the HICANN.
Specifically the vector extension allows for either use of 8 element vectors with element being halfword (1 halfword = 2 bytes) sized or 16 element vectors with each element byte sized
Thus every vector is 16 bytes or 128 Bits long.
This is also the size of each vector register that is available, 32 in total, in contrast to 32 general purpose registers with 32 bit each.
Also there is an accumulator featured as part of the vector extension which is used in every arithmetic operation and a condition register which holds 3 bits, that determine which condition applies, for every half byte or nibble, making 96 bit in total.
\todo{add special regs of normal prcessro part, CC, LR, etc.}
The PPU also features 4 KiB of memory as well as access to the synapse array of the HICANN which holds up to 32x32, thus 1024 different values.
To handle the vector unit the instruction set was extended by xz \todo{count number of vector istructions.} new instructions that partly share their opcodes with existing AltiVec instructions.

\todo{theoretical gain in speed when using vector unit compared to normal instructions especially with single load and store. include PPU clock frequencies for each part in calculation.}
