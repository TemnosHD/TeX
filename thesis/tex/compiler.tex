\chapter{Compiler Structure}
\label{chapter:compiler structure}

This part explains the structure of a compiler only on a very shallow level. Therefore the reader is encouraged to read further literature as the knowledge of a compilers structure helps a lot in understanding the way a buillt-in functions works and what problems might occur.

As already hinted by the abstract, a compiler consists of a front-end and a back-end, but also a third part that is the middle-end. These three parts sit on top of each other with the front-end on top and the back-end at the bottom and pass down the program as it is translated and optimized or compiled. But communication between the parts does not go only one way and changes that are made in the back-end affect the front-end as well!
The first part of the compilation process is the translation of code which is written in some programming language into a so called Immediate Representation (IR) that looks the same for every front-end language and usually is never seen by the user. Any supported programming language (C, C++, Java…) is implemented in its own front-end that defines how the language is translated into IR. After that the IR is send to the middle-end, which generally optimizes the IR and then passes the code to the back-end. The back-end first executes further optimizations that are target-specific followed by allocatiing registers and handling relative memory. Finally the code is translated into the assembly language that is supported by the target.
After the code is compiled and emmited as an object file it is also linked, which means combining different objectfiles and and assigning absolute memory addresses to tehm. At last the binary file emmited by the linker is loaded into the memory of the processor and then can be executed.

Most Compilers that are used nowadays are built of three basic components which handle different steps in the process of converting human-readable programming language to machine-readable machine code. As does the GNU Compiler collection (GCC) which also can be seen as made of three main parts. The so called front-end, middle-end and back-end.
All three parts work more or less independently from each other and communicate over a compiler-specific „language“, which is described a the Intermediate Representation (IR). It is typically never seen by the user and exists for a fact in many different forms [reference] one of which is Register Transfer Language (RTL), which is the lowest-level IR used by GCC. It is the most interesting IR when working on a back-end and will get more attention later in this report. But in before that we need to understand the structure an functioning of a compiler in general.

The first mentioned Front-end resembles the main interaction point between the human programmer and the machine. Front-ends are usually divided by their respective programming languages such as C, C++, Java…  and have the main task of converting any programming language into unified IR, that can be passed to the middle-end in GIMPLE or GENERIC language. Therefore no matter which language you prefer, in an ideal case code, that is syntactically identical, should not differ after it is processed by the front-end. This is due to the goal of compilers such as GCC and LLVM to support as many languages and machines as reasonably possible while offering the equally good optimization and saving themselves overall work. It obvious that a single compiler for every combination of language and machine would simply not be practical, especially as the optimization taking place in the middle-end follows the same rules for pretty much any architecture.
The middle-end main task is basically this sort of optimization, and makes for the main difference between compilers as most compilers offer the same range of front-ends and back-ends. (Part about the optimizations taking place in the middle-end). After all optimizations are through, the middle-end passes IR in form of RTL to the back-end. As you can see the middle-end rarely needs to be modified except for fundamental changes in the compilers architecture such as new kinds of optimizations and „multiple memory handling“ (also Harvard-Architecture (vielleicht))
The back-end is responsible for the final steps of the compilation process as it translates the general RTL IR into specific Assembly commands. It uses some sort of table of available assembly instructions, that is provided, and finds the best fitting instructions. GCC for example uses a Lisp-like language (is this RTL?) that uses something called insns. These combine different properties with the  Assembly commands like an equivalent set of actions that are executed, the  operands and the constraints it must satisfy. These will be further explained later. The back-end also contains the the code which implements processor-specific built-in functions.
Depending on the Compiler architecture the back-end it finally emits IR to the assembler which emits the machine code in assembly or emits assembly itself. Then finally the linker links the assembly code of all program files together and substitutes the offset addresses with absolut addresses to generate the final machine code.
reload
register spilling
register handling
endianess
wordsize

different parts of an instruction, mention: opcodes, asm instruction, operation, operand, insn, IR, builtin function/intrinsic
