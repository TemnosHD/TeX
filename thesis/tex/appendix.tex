\addcontentsline{toc}{chapter}{Appendix}
\chapter*{Appendix}

\section*{Acronyms}
\begin{acronym}
    \acro{ADC}{Analogue Digital Converter}
    \acro{ALU}{Arithmetic Logic Unit}
    \acro{ASIP}{Application Specific Instruction set Processor}
    \acro{asm}{Assembly}
    \acro{CISC}{Complex Instruction Set Computer}
    \acro{CPU}{Central Processing Unit}
    \acro{CR}{Conditional Register}
    \acro{DLS}{Digital Learning System}
    \acro{DRAM}{Dynamic RAM}
    \acro{insn}{instruction}
    \acro{IR}{Intermediate Representation}
    \acro{FPGA}{Field Programmable Gate Array}
    \acro{FPR}{Floating Point Register}
    \acro{FPU}{Ploating Point Unit}
    \acro{GCC}{GNU Compiler Collection}
    \acro{GPP}{General Purpose Processor}
    \acro{GPR}{General Purpose Register}
    \acro{HICANN}{High Input Count Neural Network}
    \acro{ISA}{Instruction Set Architecture}
    \acro{LLVM}{Low Level Virtual Machine}
    \acro{LR}{Linker Register}
    \acro{LRA}{Local Register Allocator}
    \acro{MMU}{Memory Management Unit}
    \acro{MSB}{Most Significant Bit}
    \acro{nux}{alternative name for PPU}
    \acro{POWER}{Performance Optimization With Enhanced RISC}
    \acro{PPU}{Plasiticty Processing Unit}
    \acro{RAM}{Random Access Memory}
    \acro{RF}{Register File}
    \acro{RTL}{Register Transfer Language}
    \acro{RISC}{Reduced Instruction Set Computer}
    \acro{rs6000}{RISC system/6000}
    \acro{s2pp}{synaptic plasiticity processor a.k.a PPU}
    \acro{SIMD}{Single Input Multiple Data}
    \acro{SPR}{Special Purpose Register}
    \acro{SRAM}{Static RAM}
    \acro{VE}{Vector Extension}
    \acro{VR}{Vector Register}
    \acro{VRF}{Vector Register File}
    \acro{VMX}{Vector Media eXtension}

\end{acronym}

\section*{List of nux Intrinsics}

\begin{table}
    \adjustbox{max width=\textwidth}{%
        \begin{tabular}{c c¡c¡c¡c¡c¡p{5cm}}
            \multicolumn{1}{c}{intrinsic name} & \multicolumn{1}{c}{use} & \multicolumn{4}{c}{data types} & \multicolumn{1}{c}{effect}\\
            \cline{3-6}
             & & d & a & b & c & \\
            \hline \hline
                \begin{tabular}[x]{@{}c@{}}|fxv_add|\\|vec_add|\end{tabular} & |d = fxv_add(a,b)| & same as |a| & 
                \begin{tabular}[x]{@{}c@{}} |vector signed char|\\
                                            |vector unsigned char|\\
                                            |vector signed short|\\
                                            |vector unsigned short|\end{tabular}
                                            & same as |a| & &  add |a| and |b| modulo element-wise and write the result in |d|\\ 
            \cline{3-6}
                \begin{tabular}[x]{@{}c@{}}|fxv_sub|\\|vec_sub|\end{tabular} & |d = fxv_sub(a,b)| & same as |a| & 
                \begin{tabular}[x]{@{}c@{}} |vector signed char|\\
                                            |vector unsigned char|\\
                                            |vector signed short|\\
                                            |vector unsigned short|\end{tabular}
                                            & same as |a| & &  subtract |b| from |a| modulo element-wise and write the result in |d|\\ 
            \cline{3-6}
                \begin{tabular}[x]{@{}c@{}}|fxv_mul|\\|vec_mul|\end{tabular} & |d = fxv_mul(a,b)| & same as |a| & 
                \begin{tabular}[x]{@{}c@{}} |vector signed char|\\
                                            |vector unsigned char|\\
                                            |vector signed short|\\
                                            |vector unsigned short|\end{tabular}
                                            & same as |a| & &  multiply |a| and |b| modulo element-wise and write the result in |d|\\ 
            \cline{3-6}
                \begin{tabular}[x]{@{}c@{}}|fxv_addfs|\end{tabular} & |d = fxv_addfs(a,b)| & same as |a| & 
                \begin{tabular}[x]{@{}c@{}} |vector signed char|\\
                                            |vector unsigned char|\\
                                            |vector signed short|\\
                                            |vector unsigned short|\end{tabular}
                                            & same as |a| & &  add |a| and |b| saturational element-wise and write the result in |d|\\ 
            \cline{3-6}
                \begin{tabular}[x]{@{}c@{}}|fxv_subfs|\end{tabular} & |d = fxv_subfs(a,b)| & same as |a| & 
                \begin{tabular}[x]{@{}c@{}} |vector signed char|\\
                                            |vector unsigned char|\\
                                            |vector signed short|\\
                                            |vector unsigned short|\end{tabular}
                                            & same as |a| & &  subtract |b| from |a| saturational element-wise and write the result in |d|\\ 
            \cline{3-6}
                \begin{tabular}[x]{@{}c@{}}|fxv_mulfs|\end{tabular} & |d = fxv_mulfs(a,b)| & same as |a| & 
                \begin{tabular}[x]{@{}c@{}} |vector signed char|\\
                                            |vector unsigned char|\\
                                            |vector signed short|\\
                                            |vector unsigned short|\end{tabular}
                                            & same as |a| & &  multiply |a| and |b| saturational element-wise and write the result in |d|\\ 
            \cline{3-6}
                \begin{tabular}[x]{@{}c@{}}|fxv_stax|\\|vec_st|\end{tabular} & |fxv_stax(a,b,c)| & & 
                \begin{tabular}[x]{@{}c@{}} |vector signed char|\\\\
                                            |vector unsigned char|\\\\
                                            |vector signed short|\\\\
                                            |vector unsigned short|\\\end{tabular}
                                            & |int| &
                \begin{tabular}[x]{@{}c@{}} |vector signed char*|\\
                                            |signed char*|\\
                                            |vector unsigned char*|\\
                                            |unsigned char*|\\
                                            |vector signed short*|\\
                                            |signed short*|\\
                                            |vector unsigned short*|\\
                                            |unsigned short|\end{tabular}
                                            &  |a| is stored to memory address |c + b|\\ 
            \cline{3-6}
                \begin{tabular}[x]{@{}c@{}}|fxv_outx|\end{tabular} & |fxv_outx(a,b,c)| & & 
                \begin{tabular}[x]{@{}c@{}} |vector signed char|\\\\
                                            |vector unsigned char|\\\\
                                            |vector signed short|\\\\
                                            |vector unsigned short|\\\end{tabular}
                                            & |int| &
                \begin{tabular}[x]{@{}c@{}} |vector signed char*|\\
                                            |signed char*|\\
                                            |vector unsigned char*|\\
                                            |unsigned char*|\\
                                            |vector signed short*|\\
                                            |signed short*|\\
                                            |vector unsigned short*|\\
                                            |unsigned short|\end{tabular}
                                            &  |a| is stored to synaptic address |c + b|\\ 
            \cline{3-6}
                \begin{tabular}[x]{@{}c@{}}|fxv_lax|\\|vec_ld|\end{tabular} & |d = fxv_lax(a,b)| & 
                \begin{tabular}[x]{@{}c@{}} |vector signed char|\\\\
                                            |vector unsigned char|\\\\
                                            |vector signed short|\\\\
                                            |vector unsigned short|\\\end{tabular}
                                            & |int| &
                \begin{tabular}[x]{@{}c@{}} |vector signed char*|\\
                                            |signed char*|\\
                                            |vector unsigned char*|\\
                                            |unsigned char*|\\
                                            |vector signed short*|\\
                                            |signed short*|\\
                                            |vector unsigned short*|\\
                                            |unsigned short|\end{tabular}
                                            & & |d| is read from memory address |a + b|\\ 
            \cline{3-6}
                \begin{tabular}[x]{@{}c@{}}|fxv_inx|\end{tabular} & |d = fxv_inx(a,b)| & 
                \begin{tabular}[x]{@{}c@{}} |vector signed char|\\\\
                                            |vector unsigned char|\\\\
                                            |vector signed short|\\\\
                                            |vector unsigned short|\\\end{tabular}
                                            & |int| &
                \begin{tabular}[x]{@{}c@{}} |vector signed char*|\\
                                            |signed char*|\\
                                            |vector unsigned char*|\\
                                            |unsigned char*|\\
                                            |vector signed short*|\\
                                            |signed short*|\\
                                            |vector unsigned short*|\\
                                            |unsigned short|\end{tabular}
                                            & & |d| is read from synaptic address |a + b|\\ 
            \cline{3-6}
                \begin{tabular}[x]{@{}c@{}}|fxv_sel|\end{tabular} & |d = fxv_sel(a,b,c)| & same as |a| & 
                \begin{tabular}[x]{@{}c@{}} |vector signed char|\\
                                            |vector unsigned char|\\
                                            |vector signed short|\\
                                            |vector unsigned short|\end{tabular}
                                            & same as |a| & |2-bit int| & select element from |a| if |c| applies for that index otherwise select |b|, store the result in |d|\\ 
            \cline{3-6}
                \begin{tabular}[x]{@{}c@{}}|vec_extract|\end{tabular} & |d = vec_exctract(a,b)| & 
                \begin{tabular}[x]{@{}c@{}} |signed char|\\
                                            |unsigned char|\\
                                            |signed short|\\
                                            |unsigned short|\end{tabular}
                                            &
                \begin{tabular}[x]{@{}c@{}} |vector signed char|\\
                                            |vector unsigned char|\\
                                            |vector signed short|\\
                                            |vector unsigned short|\end{tabular}
                                            & |int| & & |d| is read from synaptic address |a + b|\\ 
                 \\ vecinsert \\ vecpromote \\ vecsh fxvsh \\ fxvpcku \\ fxvpckl \\ fxvupckl \\fxvupckr \\ vecsplats16 fxvsplatb \\ fxvsplath vecsplats8 \\ fxvcmp \\ fxvmtac \\fxvmtacf \\fxvaddactacm \\fxvaddactacf\\fxvaddacm\\fxvaddacfs\\fxvmam\\fxvmafs\\fxvmatacm\\fxvmatacfs\\fxvmultacm\\fxvmultacfs\\fxvaddtac


        \end{tabular}
    }
\end{table}


\begin{tabular}{l l p{9cm}}
    mnemonic & operands & description \\
    \hline
    |add| & |RT, RA, RB| & \textbf{add} |RB| to |RA| and store the result in |RT| \\
    |addi| & |RT, RA, SI| & \textbf{add} |SI| to |RA| and store the result in |RT| \\
    |addis| & |RT, RA, SI| & \textbf{add} |SI| \textbf{s}hifted left by 16 bit to |RA| and store the result in |RT| \\
    |and| & |RA, RS, RB| & |RS| and |RB| are \textbf{and}ed and the result is stored in |RT| \\
    |b| & |target_addr| & \textbf{b}ranch to the code at |target_addr|\\
    |ble| & |BF, target_addr| & \textbf{b}ranch to the code at |target_addr| if |BF| is \textbf{l}ess or \textbf{e}qual \\
    |blr| & & \textbf{b}ranch to the code at address in the \textbf{l}inker \textbf{r}egister \\
    |cmp| & |BF, L, RA, RB| & |RA| and |RB| are \textbf{c}o\textbf{mp}ared and the result (|gt,lt,eq|) is stored in |BF|, |L| depicts if 32-bit or 64-bit are compared \\
    |cmplwi| & |BF, RA, SI| & |RA| \textbf{c}o\textbf{mp}ared \textbf{l}ogically \textbf{w}ordwise with \textbf{i}mmediate |SI| and the result is stored in |BF|\\
    |and| & |RA, RS, RB| & |RS| and |RB| are \textbf{and}ed and the result is stored in |RT| \\
    |eieio| & & enforce in-order execution of I/O\\
    |isync| & & instruction cache synchronize\\
    |la| & |RT, D(RA)| & \textbf{l}oad \textbf{a}ggregate |D + RA| into |RT|\\
    |li| & |RT, SI| & \textbf{l}oad \textbf{i}mmediate value |SI| into |RT|\\
    |lis| & |RT, SI| & \textbf{l}oad \textbf{i}mmediate value |SI| \textbf{s}hifted left by 16 bit into |RT|\\
    |lbz| & |RT, D(RA)| & \textbf{l}oad \textbf{b}yte at address |D+RA| into |RT|, fill the other bits with \textbf{z}eros \\
    |lwz| & |RT, D(RA)| & \textbf{l}oad \textbf{w}ord at address |D+RA| into |RT|, fill the other bits with \textbf{z}eros \\
    |mflr| & |RT| & \textbf{m}ove \textbf{f}rom \textbf{l}inker \textbf{r}egister to |RT|\\
    |mr| & |RT, RA| & \textbf{m}ove \textbf{r}egister |RA| to |RT| \\
    |nop| & & halts execution until the previous instructions are finished EDITHIS\\
    |rlwinm| & |RA, RS, SH, MB, ME| & \textbf{r}otate \textbf{l}eft \textbf{w}ord in |RS| by \textbf{i}mmediate |SH| bits then a\textbf{n}d with \textbf{m}ask which is 1 from |MB+32| to |ME+32| and 0 else, store to |RA|\\
    |stw| & |RS, D(RA)| & \textbf{st}ore \textbf{w}ord from |RS| to address |D+RA|\\
    |stwu| & |RS, D(RA)| & \textbf{st}ore \textbf{w}ord from |RS| to address |D+RA| and \textbf{u}pdate RA to |D+RA|\\
    |sync| & & synchronize data cache\\
\end{tabular}
\add{caption and reference to PPC book and asm website correct the table}

Obviously the mnemonics follow a certain pattern that has letters which can be interchanged to alter the meaning of the mnemonic, some of these characters are:
\begin{description}
    \lstitem{i} indicates that the instructions uses an immediate value
    \lstitem{b} stands for byte and references the size of the operand
    \lstitem{h} stands for halfword and references the size of the operand
    \lstitem{w} stands for word and references the size of the operand
    \lstitem{s} indicates that one of the operands is shifted
    \lstitem{g, ge, l, le, e} stand for greater, greater or equal, less, less or equal and equal which is the possible content of the conditional register
\end{description}

There are also special operands which might occur in |asm| which behave like pointers:
\begin{description}
    \lstitem{@l(C)} is equivalent to the lower order 16 bits of of |C| in the symbol table
    \lstitem{@ha(C)} is equivalent to the higher order 16 bits of of |C| in the symbol table and minds the sign extension
\end{description}

Additionally there exist markers which are intended for debugging:
\begin{description}
        \lstitem{.loc # # #} marks a line of code (file, line, column) in the source file
        \lstitem{.LVL} is a local label which can be discarded
        \lstitem{.LFB} marks the begin of a function
        \lstitem{.LFE} marks the end of a function
        \lstitem{.LC0} is a constant of the literals table at position |0|
\end{description}

\add{all these tables in the appendix}
\newpage
\listoftodos[Notes]

