\chapter{Fundamentals and Applications of Computer Architectures and Compiler Design}
\label{chapter:methods}

\section{Hardware Implementation of Neural Networks}
As this thesis mainly focuses on a processor that is an essential part of the \textbf{HICANN-DLS} (High Input Count Analogue Neural Network-Digital Learning System) we will first talk about the HICANN-DLS as a whole and then look into the \textbf{PPU} (Plasticity Processing Unit) in detail while also addressing processor architecture in general.

The HICANN-DLS was built to emulate neural networks at high speeds with low power consumption.

On a very abstract level neurons in the brain resemble nodes in a network.
Neurons are interconnected through dendrites, synapses and axons which can be of different coupling strength. 
In contrast to many other systems \reference{other systems} we use neuron model that is \textbf{spike based}.
This means that a neuron is activated only for a short continuous time, called a spike, and sends out this spike through its axon to neurons that are connected via synapses.
Between those spikes the neuron is resting and not sending any signals.
Synapses can work quite differently but have in common that there is a certain weight associated to them, which we will call synaptic weight.
This is equal to a gain with which the pre-synaptic signal is either amplified or attenuated.
The signal is then passed through the dendrite of the post-synaptic neuron to the soma where all incoming signals are integrated.
If the integrated signals reach a certain threshold the neuron spikes and sends this signal to other neurons~\citep{silbernagl2009color}.

\add{bild synaptic array}
\begin{figure}[htpb]
    \centering
    %
\noindent{\begin{minipage}{\textwidth}
   \begin{tikzpicture}[style={circle,draw,fill=black,minimum size=15}, line width=.2mm,
            vblock/.style={
            draw,
            fill=white,
            rectangle,
            minimum width=1.0cm,
            minimum height=0.3,
            font=\tiny},
            hblock/.style={
            draw,
            fill=white,
            rectangle,
            minimum width=.3,
            minimum height=1.0cm,
            font=\tiny}]

           \foreach \y in {0,...,31} 
           {
                      \node(domain)[vblock] (inrec\y) at (0, 0.3*\y + 1) {}; 
                      \draw[-] (inrec\y.north east) -- ($(inrec\y.north east)!.5!(inrec\y.south east)+(0.3,0)$) -- (inrec\y.south east);
                      \node (in\y) at (0, 0.3*\y + 1) {\tiny \y}; 
                      \draw[-, red] ($(inrec\y.north east)!.5!(inrec\y.south east)+(0.3,0)$) -- (0.3*31 + 1,0.3*\y + 1);
                  }
       \foreach \x in {0,...,31}
       { 
           \node(domain)[hblock] (neurec\x) at (0.3*\x + 1, 0) {}; 
           \node[rotate=90,] (neu\x) at (0.3*\x + 1, 0) {\tiny \x}; 
           \draw[-, blue] (neurec\x.north) -- (0.3*\x + 1,0.3*31 + 1);
           \foreach \y in {0,...,31} 
                  {\pgfmathtruncatemacro{\vlabel}{\y}
                  \draw [black,fill=black,radius=0.5mm, minimum size=0]  (0.3*\x + 1,0.3*\y + 1) circle;}
              }

              \node[rotate=90] (input) at (-1, 0.3*31/2 + 1) {synaptic input};
              \node[] (input) at (0.3*31/2 + 1, -1) {neurons};
   \end{tikzpicture}
\end{minipage}}

    \caption{\label{fig:array} structure of the synaptic array}
\end{figure}

The HICANN-DLS system implements a simplified neural network model in analogue electronics in order to emulate neuronal networks in a biologically plausible parameter range.

At its core it has a so called \textbf{synaptic array} that connects 32 neurons which are located on a single chip to 32 different pre-synaptic inputs.
Pre-synaptic inputs are arranged along one side of the chip which is orthogonal to the arrangement of the neurons.
They enclose a 2D field which will be our synaptic array as it mainly consists of synapse circuits.
All neurons reach into the array through conductors that are organized in columns.
The pre-synaptic inputs respectively have conductors that resemble rows in the array.
At each intersection of of those conducting lines, a synapse is placed that thereby connects a neuron and a pre-synaptic input.
Overall this gives 1024 synapses that interconnect every neuron and every synapse.

A \textbf{field programmable gate array} (FPGA) is connected to all pre-synaptic inputs and feeds external spikes into the ship which can either be computed by the FPGA or routed spikes from another ship.
Along an input row the signal of one pre-synaptic input reaches all synapses that connected and is processed individually.
Each synapse holds a 6-bit wide \textbf{synaptic weight} and an internal decoder address, which both can be changed from outside.

The FPGA sends a 6-bit address whenever it sends a spike to a pre-synaptic input which then is compared by each synapse to the addresses they hold themselves.
In case the addresses match, each synapse multiplies an output signal with the weight it stores and sends the result along a column where is reaches a neuron.
All signals that are sent, are collected along a column and reach the neuron.
Inside the neurons the resulting current input is evaluated in regard to a threshold and other parameters which decide on whether the neuron is spiking or not.
If the neuron is spiking it sends an output signal to the FPGA which is responsible for spike routing in the first place.
All of this is done continuously and may not follow discrete time steps.
Along each column sits a \textbf{correlation ADC} (CADC) that converts the signal one neuron receives into digital data for analysis but can also be accessed by the PPU similar to the synaptic array.

The HICANN-DLS is also equipped with a processing unit that includes a vector extension and some memory for it to operate on.
This is the plasticity processing unit (PPU) which is also connected to the synapse array for it to read and write synaptic weights.

The will use the following naming convention throughout this thesis:
\begin{description}
    \item[Plasticity Processing Unit PPU] is the processor which is part of HICANN-DLS and mainly responsible for plasticity of synapses.
    \item[nux] refers to the complete architecture of the PPU together with its vector extension.
    \item[s2pp] is short for synaptic plasticity processor but we will use this as name of the vector extension.
\end{description}

Synapses in the synapse array are realized as small repetitive circuits that contain 8 bits of accessible information each although only 6 bits are used as weights and the spare two upper bits are used for calibration.
The synapse array can also be used in 16-bit mode for higher accuracy that combines the weights of two synapses to a 12-bit weight.
\add{figure synapse}

Digital configuration of the synapses and writing PPU programs to the memory is handled by an FPGA that has access to every interface of a HICANN-DLS chip.

It was developed to handle plasticity and as such applies different plasticity rules to synapses during or in between experiments.
This is done much faster by the PPU than by the FPGA which is important for achieving experimental speeds that are $10^{3}$ times faster than their biological counterparts.
In general the PPU is meant to handle plasticity of the synapses during experiments while the FPGA should be used to initially set up an experiment, manage spike input and record data 

\section{Processor Architectures and the Plasticity Processing Unit}
Although the main goal of this project is to provide an alternative analogue architecture like the HICANN-DLS does, there are advantages to classic computing which are needed for some applications and almost all contemporary processors are built using the so called von-Neumann architecture \todo{add reference}.

The main advantage of digital systems over analogue systems, such as the human brain, is the ability to do numeric and logical operations at much higher speeds and precision as well as the availability of existing digital interfaces.
For this reason ``normal'' processors are responsible for handling experiment data as well as configuration of an experiment in the HICANN-DLS.
We will now shortly get in touch with basics of such processors and explain common terms while referring to the PPU at times when it is convenient.
\\
The PPU, which was designed by \citep{PPU}, is a custom processor, that is based on the Power Instruction Set Architecture (PowerISA), which was developed by IBM since 1990. 
Specifically the PPU uses POWER7 which was released in 2010 as a successor of the original POWER architecture and runs at 100 MHz clock frequency.
\\
\add{figure classic architecture}
%general stuff
In general a microprocessor can be seen as a combination of two units which are an operational section and a control section.
The control section is responsible for fetching instructions and operands, interpreting them, controlling their execution as well as reading and writing to the main memory or other buses.
The operational section on the other hand saves operands and results only as long as they are needed and performs logic or arithmetic operation on these as instructed by the control section.
Prominent parts of the operational section are the \textbf{arithmetic logic unit} (ALU) and the \textbf{register file} (RF).

%register
The register file can be seen as short-term memory of the processor.
It consists of several repeated elements, called \textbf{registers}, that save data and have share the same size which is determined by the architecture; the 32-bit architecture of nux for instance has 32-bit wide registers.

Typically the number and purpose of registers varies for different architectures.
Common purposes of registers are:
\begin{description}
    \item[general-purpose register GPR] These registers can store values for various causes but in most cases are soon to be used by the ALU. Most registers on a processor are typically GPRs.
        Any register that is not a GPR is called a \textbf{special purpose register SPR}
    \item[link register LR] This register marks the jump point of function calls. After a function completes, the program jumps to the address in the link register.
    \item[condition register CR] This register's value is set by an instruction that compares one or two values in GPRs. Its value can be a condition for some instructions if they are to be executed or not.
\end{description}        
The ALU uses values which are stored in the register file to perform the aforementioned logic or arithmetic operations and saves the results there as well.

Some architectures also have an accumulator that is often part of the ALU.
Intermediate results can be stored there because access to the accumulator is the fastest possible but it can only holds a single value at a time.
\\
\\
\begin{figure}[htpb]
    \centering
    \begin{bytefield}[endianness=little]{32}
        \bitheader{0,1,7,15,31}\\
        \colorbitbox{lightgray}{1}{\tiny bit} && \bitbox{31}{}\\
        \colorbitbox{lightgray}{8}{byte} && \bitbox{24}{}\\
        \colorbitbox{lightgray}{16}{half-word} && \bitbox{16}{}\\
        \colorbitbox{lightgray}{32}{word}\\
    \end{bytefield}
    \caption{\label{fig:bitlength} illustration of word sizes for 32-bit words}
\end{figure}

%memory
Memory is often implemented as \textbf{random access memory} which is both readable and writable it can be split into two popular types which are \textbf{static random access memory} (SRAM) and \textbf{dynamic random access memory} (DRAM).
They differ in how bits are set in RAM.
SRAM uses Flip Flops to switch transistors that indicate which bit is set, while DRAM uses capacities that are charged to do so.\unsure{should I keep this}
Memory of a von-Neumann machine contains both, the program and data and is often seen as blocks with addresses.
Because the smallest amount of information which we are interested in is usually a byte, each address is equivalent to one byte in memory.
A program is simply a list of instructions in memory that belong to the \textbf{instruction set}.
Each instruction thereby is a specific sequence of bits. 
\begin{figure}[htpb]
    \centering
    \begin{bytefield}[endianness=big, bitwidth=0.027777\linewidth]{32}
        \bitheader{0,7,15,23,31}\\
        \bitboxes{8}{{opcode}{operand 0}{operand 1}{operand 2}}
    \end{bytefield}
    \caption{\label{fig:opcode} representation of machine instruction in memory}
\end{figure}

%instructions
An instruction such as \ref{fig:opcode} is called a \textbf{machine isntruction} and combines several elements.
The first part is called an \textbf{opcode} and is typically an 8-bit number that identifies an operation.
The ALU reads this number and performs a set of so called \textbf{micro instructions} accordingly.

During a single clock cycle a chip can perform a single micro instruction.
An example for micro instructions in an add instruction (|d = add(a,b)|) would be:
\begin{lstlisting}[caption=example of micro instructions, label=lst:microinstruction]
    1. fetch instruction from memory
    2. decode instruction
    3. fetch first operand a
    4. fetch second operand b
    5. perform operation on operands
    6. store result
\end{lstlisting}\todo{inhaltlicher zusammenhang}

Opcodes are often represented by an alias string such as |add| that is called a \textbf{mnemonic}.
The opcode is followed by several addresses that refer to the location a value can be taken from or where it should be stored.
These addresses are called \textbf{operands} and can either be a memory address or a register number.

It takes up to several hundred cycles for instructions to access memory which effectively stalls the processor until the memory instruction has finished and reduces performance.

\begin{equation*}
    \text{speed(accumulator)} > \text{speed(register)} \gg \text{speed(memory)}
\end{equation*}
Therefore a user should try to avoid memory access as much as possible and use registers instead.

%instruction set
As we mentioned the instruction set before, the complexity of an instruction set is also very important for performance. 
A complex instruction set might seem favorable to increase performance at first but a smaller instruction set also bears some advantages.
Because every instruction has to be represented by a circuit in the ALU, a smaller instruction set safes space and is easier to design.
In general developing a processor architecture is always a trade-off between factors like: available chip space, instruction set and design complexity, energy consumption and maximum clock frequency.
Because of this, processors can be classified in two main groups:
\begin{description}
    \item[CISC] Complex Instruction Set Computer
    \item[RISC] Reduced Instruction Set Computer
\end{description}
The latter usually has an instruction set that is reduced to simple instructions as |add| or |sub| and connects these to create more complex instructions.
It also can be operated at higher clock frequencies, therefore it is the perfect architecture for applications that need to do simple arithmetic as fast as possible.

The PPU is a RISC architecture so we will focus more on RISC's key features throughout this chapter.

%assembly
The complexity of an instruction set also affects developers directly, if they are working on low-level code.
Programs that only consist of machine instructions a called \textbf{assembly} which is the lowest level of representation of a program that is still is human-readable.
Assembly instructions follow the same scheme as machine instructions do:
\begin{figure}[htpb]
    \centering
    \begin{bytefield}[endianness=big, bitwidth=0.027777\linewidth]{32}
        \bitheader{0,7,15,23,31}\\
        \bitboxes{8}{{mnemonic}{operand}{operand}{operand}}\\
        \bitheader{0,7,15,23,31}\\
        \bitboxes{8}{{add}{r1 \\ \tiny register address}{0x3000 \\ \tiny memory address}{5 \\ \tiny immediate operand}}
    \end{bytefield}
    \caption{\label{fig:opcode} representation of machine instruction in memory}
\end{figure}
\begin{lstlisting}
    add r1, 0x3000, 5
    mnemonic operand/result operand operand
\end{lstlisting}
In RISC architectures instructions typically consist of 3 operands because operations are usually between registers only (except for load/store memory instructions).
The mnemonic in most cases is a named after the first letters of the instructions full name, which is emphasized in the following table.
The operand can be of three different types which are all shown above.
They either represent a specific register (r1 = register 1), a memory address (0x3000 = the value at memory location 0x3000) or an immediate value (5 = the integer 5).
Register operands can also have an indirect use, which means that that content of the register is taken into account.
I.e. a memory address can be saved to the register and an operation uses the value at the memory location which the register refers to.
This is often the case for RISC architectures as they support only one \textbf{load} and one \textbf{store} instruction, that either loads a value from memory into a register, or stores a register's value in memory.
Therefore RISC architectures often qualify as load/store architectures as the memory address is first stored in a register that afterwards is an operand of a memory instruction.
This is called an indirect operand.

As address size is limited by the register, the maximum amount of memory that can be used is:
\begin{equation}
    2^{n} byte \xrightarrow{\text{n = 32 bit}} 2^{32} byte \approx 4*10^{9} byte = 4 GiB
\end{equation}

An overview of different assembly mnemonics for the POWER7 ISA can be seen in the Appendix as table~\ref{table:asm}.
It is accompanied by tables~\ref{table:letters} and \ref{table:asmmisc} which give further information on assembly.

%RISC
Many RISC architectures have an instruction set that consists exclusively of 3 operand instructions and any instructions that seem to have less than that, are just mapped to more complex instructions that have three operands but still need only the minimum amount of clock cycles.
Such simple instructions are similar to micro instructions, which were mentioned earlier, and every simple instruction has the same low number of micro instructions.
RISC architectures therefore start ``pipelining'' instructions, which means starting the next machine instruction as the previous machine instruction just performed the first step in a clock cycle.
Ideally, this will increase the performance by a factor that is equal to the number of micro instructions in a machine instruction.

It must be noted though, that the processor has to implement detection of \textbf{hazards}, which are data dependencies between instructions; e.g. one instruction needs the result of another.
Such an instruction is then postponed to a delay-slot and other instructions that do not cause hazards are executed instead.
The result is reordering of instructions on a processor level.
\\
\\
Processors sometimes have so called \textbf{co-processors} for complex instructions that are not included in the instruction set but are still needed.
An example would be multiplication on most RISCs, which would need many cycles when split in |add| instructions but a co-processor can perform this in just a few cycles.
In such a case the control section recognizes the |mult| opcode and passes it to the co-processor instead of the ALU.

This can be extended to whole units similar to the ALU existing in parallel.
One example would be a \textbf{floating point unit} (FPU) which is nowadays standard for most processors and handles all instructions on floating point numbers.
For this the FPU has its own \textbf{floating point registers} (FPRs) in a separate register file on which it performs instructions and which also has access to the memory.\unsure{keep this?}

An different kind of extension are vector extensions that do the same as the FPU but for vectors instead of floats.
This is mostly wanted for highly parallel processes such as graphic rendering or audio and video processing \todo{reference}.
Early supercomputers such as the Cray-1 \todo{reference} also made use of vector processing to gain performance by operating on multiple values  simultaneously through a single register.
This could either be realized through a fully parallel architecture or more easily through pipelining instructions for vector elements.
The latter one is possible since there are typically no dependencies, hence no hazards, between single elements in the same vector.
Nowadays basically all common architectures support vector processing.
A few examples are:

\noindent\begin{minipage}{\textwidth}
    \vspace{1em}
    \begin{minipage}{0.4\textwidth}
\begin{itemize}
    \item x86 with SSE-series and AVX
    \item IA-32 with MMX
    \item AMD K6-2 with 3DNow!
\end{itemize}
\end{minipage}
    \begin{minipage}{0.6\textwidth}
\begin{itemize}
    \item PowerPC with AltiVec and SPE
    \item ARM with NEON
\end{itemize}
\end{minipage}
    \vspace{1em}
\end{minipage}

The s2pp \textbf{vector extension} (VE) of nux is the PPU's distinct feature that allows for Single Input Multiple Data (SIMD) operations on synaptic weights.
The VE is only weakly coupled to the general purpose processor (GPP) of the PPU nand both parts can operate in parallel while interaction is highly limited.
To handle the vector unit the instruction set was extended by 53 new vector instructions that partly share their opcodes with existing AltiVec instructions.
This renders no problem since the nux does not recognize AltiVec opcodes and most like is not going to in the future.
The vector register file includes (VRF) 32 new vector registers which are each 128-bit wide~\citep{AV:registers}.
This allows for either use of vectors with 8 halfword sized elements or 16 byte sized elements which are 128 bits long as seen in figure~\ref{fig:vectorsize}.

But we take also a special interest in the AltiVec vector extension itself which was developed by Apple, IBM and Motorola in the mid 1990's and is also known as Vector Media Extension (VMX) and Velocity Engine for the POWER architecture. 
The AltiVec extension provides a similar single-precision floating point and integer SIMD instruction set.
Its vector registers can either hold sixteen 8-bit |char|s (V16QI), eight 16-bit |short|s (V8HI), four 32-bit |int|s (V4SI) or single precision |float|s (V4SF), each signed and unsigned.
\begin{figure}[htpb]
    \centering
    \begin{bytefield}[endianness=little, bitwidth=\widthof{\tiny Integer~}/8]{128}
        \bitheader{0,7,15,31,63,127}\\
        \begin{rightwordgroup}{V16QI}\bitboxes{8}{{QI}{\tiny Quarter \\ Integer}{}{}{}{}{}{}{}{}{}{}{}{}{}{}}\end{rightwordgroup}\\
        \bitheader{0,7,15,31,63,127}\\
        \begin{rightwordgroup}{V8HI}\bitboxes{16}{{HI}{\tiny Half \\ Integer}{}{}{}{}{}{}}\end{rightwordgroup}\\
        \bitheader{0,7,15,31,63,127}\\
        \begin{rightwordgroup}{V4SI}\bitboxes{32}{{SI}{\tiny Single \\ Integer}{}{}}\end{rightwordgroup}\\
        \bitheader{0,7,15,31,63,127}\\
        \begin{rightwordgroup}{V4SF}\bitboxes{32}{{SF}{\tiny Single \\ Float}{}{}}\end{rightwordgroup}\\
    \end{bytefield}
    \caption{\label{fig:vectorlength} different vector structures}
\end{figure}

It resembles most characteristics of the s2pp vector extension, like a similar VRF, and is already implemented in the PowerPC back-end of GCC.
There are a few differences though:
\begin{addmargin}[2em]{0em}
    The s2pp VE features a vector accumulator of 128 bits width and a vector condition register (VCR) which holds 3 bits for each half byte of the vector, making 96 bit in total.
    Instructions on the s2pp VE can be specified to operate only on those elements of a vector, that meet the condition in the corresponding bits in the VCR, while the AltiVec VE utilizes the CR of the PowerPC architecture.
    This does not allow for selective operations on individual elements through the CR, but allows for checking if all elements meet the condition in a single instruction.
    If element-wise selection is needed, AltiVec offers this through vector masks.
    
    The AltiVec VE has two register on its own though, which are the VCSR and VRSAVE registers.
            The Vector Status and Control Register (VSCR) is responsible for detecting saturation in vector operations and decides which floating point mode is used.
            The Vector Save/Restore Register (VRSAVE) assists applications and operation systems by indicating for each VR if it is currently used by a process and thus must be restored in case of an interrupt.
    
    Both of these registers are not available in the s2pp VE but would likely not be needed for simple such arithmetic tasks which the PPU is meant to perform.
\end{addmargin}

We already stated that all instructions of VEs must first pass the control unit, which detects vector instructions and then passes them to the VE.
These instructions then go into an instruction cache for vector instructions where they are fetched from in-order.
On nux the instructions then shortly stay in a reservation station that is specific for each kind of operation an thus allows for little out-of-order operation for instructions in these reservations stations.\add{figure vector extension}
This allows for performing some arithmetic operations on a vector during the process of accessing a different vector in memory.
The result is faster processing speed as pipelining for each instruction is also supported.
The limiting factor for this though remains the VRF's single port for reading and writing.
A more limiting factor in comparison is the shared memory interface of s2pp and GPP.

Normally processors themself do not keep track of memory directly.
This is usually done by a \textbf{memory management unit} (MMU) or a \textbf{memory controller} (MC).
It handles memory access by the processor and can provide a set of virtual memory addresses which it translates into physical addresses.
Most modern MMUs also incorporate a cache that stores memory instructions while another memory instruciton is performed.
It also detects dependencies within this cache, which can be resolved there.
Ultimately this results in faster transfer of data as two or more instructions do not need to access the same memory to fetch a value.

Not all MMUs support this though, which might lead to certain problems when handling memory.
If instructions are reordered due to pipelining while dependencies on the same memory address are not detected correctly, an instruction may write to memory before a different one could load the previous value it needed from there.
Another reason could be delayed isntructions as it was mentioned above.
For this reason exist memory barriers.

A \textbf{memory barrier} is an instruction that is equal to a guard, placed in code, that waits until all load and store instructions issued to the MMU are finished.
It therefore splits the code into instructions issued before the memory barrier and issued after the memory barrier.
This prohibits any instruction form being executed on the wrong side of the barrier due to reordering and thereby generally prevents conflicting memory instructions.

One kind of memory barrier is |sync| which does exactly what we just described.
This and other memory barriers are also described in the appendix (\ref{table:asm}).

\begin{lstlisting}[caption=The memory barrier ensures that the first store was performed before the second store is issued., label=lst:sync]
stw     r7,data1(r9)#store shared data (last)
sync    export barrier
stw     r4,lock(r3)#release lock
\end{lstlisting}

Using |sync| can result in up to a few hundred cycles of waiting for memory access to be finished and therefore this should only be done if necessary.

The PPU's memory is 16 kiB which is accompanied by 4 kiB of instruction cache.\todo{add description to instruciton cache}
Together iwth the PPU this is called the plasticity sub-system.
The MMU of this system is very simple as it does not cache memory instructions and also has matching virtual and physical addresses thus memroy barriers can become necessary at times.
\\
\\
Another feature of the PPU is the ability to read out spike counts and additional information through a bus which is accessible through the memory interface of the MMU.
It uses the upper 16 bits of a memory address for routing
These are available because the memory is only 16 kiB large which is equivalent to 16-bit addresses.
A pointer to a virtual memory address allows to read for example spike counts during an experiment.
This is done the same way for the whole chip configuration such as analog neuron parameters.

The fact, that the system views the highest as the greatest values, makes this a \textbf{big-endian} system.
\\
\\
The memory bus is also accessible by the FPGA .
This is needed for writing programs into the memory as well as getting results during or after experiments also allows for communication between a ``master'' FPGA and a ``slave'' PPU.

A \textbf{bus} is the conneciton between parts of a processor and used for data transfer.
Bus speeds are very high as they transport data in parallel such as the contents of a register.
Thus most buses should be as wide as a register of the processor.
But as buses of such width need much space, some architectures use narrower buses with fewer bits than a register and instead use two instructions to transfer the contents of a full register.
The upper half of the width is called \textbf{high order bits} and the bottom half is called \textbf{low order bits} for big-endian architectures.
Systems of this sort are described as 32/16-bit architecture, which means that registers are 32 bit wide while buses are only 16 bit wide.
As the higher order bits of registers are not as often used as the lower ones this results in less performance loss than one would initially expected.

The main feature of the PPU though is access to the synaptic array which is offerd through an extra bus.\unsure{maybe rather an isntruction not a whole bus}
Still the synapse array is mainly accessible through the main memory bus by setting the first bits similar to the spiking rate information.
Using this extra bus or the instructions associated with it is nonetheless more comfortable and gives more structure to the program but does not increase performance.
\\
\\
When using vector instruction for nux, one must always keep in mind that the weights in the synaptic array only consist of the latter 6 or 12 bits which are in a vector register and are right aligned.
\begin{figure}[htpb]
    \centering
    \begin{subfigure}[b]{\textwidth}
        \centering
    \begin{bytefield}[bitwidth=0.11111111\textwidth]{8}
        \bitheader[endianness=little,bitformatting={\tiny\hspace*{-6em}}]{0-8}\\
        \bitbox{1}{\color{lightgray}\rule{\width}{\height}} & \bitbox{1}{\color{lightgray}\rule{\width}{\height}} & \bitbox{1}{$-1$} & \bitbox{1}{$2^{-1}$} & \bitbox{1}{$2^{-2}$} & \bitbox{1}{$2^{-3}$} & \bitbox{1}{$2^{-4}$} & \bitbox{1}{$2^{-5}$}\\
    \end{bytefield}
    \caption{\label{subfig:synapse} fractional representation of weights in synapse and in vector unit}
    \end{subfigure}
    \begin{subfigure}[b]{\textwidth}
        \centering
    \begin{bytefield}[bitwidth=0.11111111\textwidth]{8}
        \bitheader[endianness=little,bitformatting={\tiny\hspace*{-6em}}]{0-8}\\
        \bitbox{1}{sign} & \bitbox{1}{$2^{-1}$} & \bitbox{1}{$2^{-2}$} & \bitbox{1}{$2^{-3}$} & \bitbox{1}{$2^{-4}$} & \bitbox{1}{$2^{-5}$} & \bitbox{1}{$2^{-5}$} & \bitbox{1}{$2^{-7}$}\\
    \end{bytefield}
    \caption{\label{subfig:fracVE} fractional representation of weights in synapse and in vector unit}
    \end{subfigure}
    \caption{\label{fig:fractional} fractional representation of weights in synapse and in vector unit}
\end{figure}

Figure~\ref{fig:fractional} displays synaptic weights as a fractional number between $-1$ and $1-2^{-5}$.
The weight is the sum of the values where the bits are set to one.
In contrast to this, the sign-bit changes its meaning when used in the vector unit and becomes a factor instead of a summand.
This is the reason why a user must shift the vector's elements when reading/writing to the synapse array, as only then do special attributes of instructions work properly.
An example would be \textbf{saturation} which predefines a minimum and maximum value and in case, the result is out-of-range, sets the value to the closest limit.
For this to work properly the bit representation must match the intended one, which is the above mentioned fractional representation, and the values must also be correctly aligned.

An overview of all opcodes is provided by \cite{nuxmanual}, which is recommended as accompanying literature besides this thesis.
In general these opcodes are divided into 6 groups of instructions:
\begin{description}
    \item[modulo halfword/byte instructions] apply a modulo operation after every instruction which causes wrap around in case of an overflow at the most significant bit position.
        Each instruction is provided as halfword (modulo $2^{16}$) and as byte instruction (modulo $2^{8}$).
    \item[saturation fractional halfword/byte instructions] allow for the results only to be in the range between $2^{-1}$ and $2^{-15}$ for halfword and $2^{-7}$ for byte instructions.
    \item[permute instructions] perform operations on vectors that handle elements of vectors only as a series of bits.
    \item[load/store instructions] move vectors between vector registers and memory or the synapse array.
\end{description}



\section{Basic Compiler Structure}
\label{section:compiler}
At its core every compiler translates a source-language into a target-language~\ref{fig:compiler}, most often it translates a high-level, human readable programming language into a machine language that consists of basic instructions that build complicated structures.
In doing so, compilers may be the essential part in everyday lives of programmers everywhere.
And since the point in time when the first compilers emerged, their development played a big role in making computers such an important part of everyday life as they are today.

\begin{figure}[htpb]
    \begin{subfigure}[c]{0.25\textwidth}
        
    \tcbset
    {enhanced,colframe=blue!70!black,colback=white!50!blue,colupper=red!50!black,
    fonttitle=\bfseries,nobeforeafter,center title, noparskip}
    \noindent{\begin{minipage}{\textwidth}
        \begin{minipage}[c][][c]{0.25\textwidth}
            \begin{tcolorbox}[tcbox raise base, width=\linewidth, remember as=cp, enhanced, watermark text=Compiler]
                \begin{tcolorbox}[enhanced, breakable, noparskip,opacityframe=0.0, opacityback=0.0, height=2cm, width=\linewidth]
                \end{tcolorbox}
                \begin{tcolorbox}[enhanced, breakable, noparskip,opacityframe=0.0, opacityback=0.0, height=1cm, width=\linewidth]
                \end{tcolorbox}
                \begin{tcolorbox}[enhanced, breakable, noparskip,opacityframe=0.0, opacityback=0.0, height=1.5cm, width=\linewidth]
                \end{tcolorbox}
            \end{tcolorbox}
            \begin{tikzpicture}[overlay,remember picture,line width=1mm]
                \draw[->, shorten >=-1.5mm] ($(cp.north)+(0,1)$) -- node [left] {program code} (cp.north);
                \draw[->] (cp.south) -- node [left] {machine files} ($(cp.south)+(0,-1)$);
            \end{tikzpicture}
        \end{minipage}
\end{minipage}}

        \caption{\label{fig:compiler} Representation of a compiler.}
    \end{subfigure}
    \begin{subfigure}[c]{0.3\textwidth}
            \tcbset
    {my box/.style={enhanced,colframe=blue!70!black,colback=white!50!blue,colupper=red!50!black,
        fonttitle=\bfseries,nobeforeafter,center title, noparskip, size=small},
    every box on layer 1/.style={every box},
    every box on layer 2/.style={reset,my box}}
\begin{tcolorbox}[enhanced jigsaw, width=\textwidth, opacityframe=0.0, opacityback=0.0]
\begin{tcolorbox}[enhanced, height=.5cm, width=\linewidth, remember as=pp, opacityframe=0.0, opacityback=0.0]\end{tcolorbox}
\begin{tcolorbox}[enhanced, height=.7cm, width=\linewidth, watermark text=Preprocessor, remember as=pp]\end{tcolorbox}
\begin{tcolorbox}[tcbox raise base, width=\linewidth, enhanced jigsaw, remember as=cmp]
    \begin{tcolorbox}[enhanced, breakable, noparskip,opacityframe=0.3, opacityback=0.3, height=1.4cm, width=\linewidth, watermark text=Front-End, remember as=fe]
    \end{tcolorbox}
    \begin{tcolorbox}[enhanced, breakable, noparskip,opacityframe=0.3, opacityback=0.3, height=0.7cm, width=\linewidth, watermark text=Middle-End, remember as=me]
    \end{tcolorbox}
    \begin{tcolorbox}[enhanced, breakable, noparskip,opacityframe=0.3, opacityback=0.3, height=1.1cm, width=\linewidth, watermark text=Back-End, remember as=be]
    \end{tcolorbox}
\end{tcolorbox}

\end{tcolorbox}

\begin{tikzpicture}[overlay,remember picture,line width=1mm]
    \draw[->, shorten >=-1.5mm] ($(pp.north)+(0,1.5)$) -- node [left] {program code} (pp.north);
    \draw[->, shorten >=-1.5mm] (pp.south) -- (fe.north);
    \draw[->, shorten >=-1.5mm] (fe.south) -- (me.north);
    \draw[->, shorten >=-1.5mm] (me.south) -- (be.north);
    \draw[->] (be.south) -- node [left] {machine files} ++(0,-1.5);
\end{tikzpicture}

        \caption{\label{fig:cmpstruct} Schematic overview of different compiler stages.}
    \end{subfigure}
    \begin{subfigure}[c]{0.4\textwidth}
            \tcbset
    {my box/.style={enhanced,colframe=blue!70!black,colback=white!50!blue,colupper=red!50!black,
        fonttitle=\bfseries,nobeforeafter,center title, noparskip, size=small},
    every box on layer 1/.style={every box},
    every box on layer 2/.style={reset,my box}}
\begin{tcolorbox}[enhanced jigsaw, width=\linewidth, remember as=pp, opacityframe=0.0, opacityback=0.0]
\begin{tcolorbox}[tcbox raise base, width=\linewidth, enhanced jigsaw, remember as=cp]
    \begin{tcolorbox}[enhanced jigsaw, breakable, noparskip,opacityframe=0.3, opacityback=0.3, width=\linewidth, size=small]
        \begin{tcolorbox}[enhanced,center title,width=\linewidth, remember as=sc, height=0.7cm,top=1mm,bottom=1mm]
            \begin{center}Scanner\end{center}
        \end{tcolorbox}
        \begin{tcolorbox}[enhanced,width=\linewidth, remember as=ps,top=1mm,bottom=1mm]
            \begin{center}Parser\end{center}  
        \end{tcolorbox}
        \begin{tcolorbox}[enhanced, width=\linewidth, remember as=sa,top=1mm,bottom=1mm]
            \begin{center}Semantic Analyzer \end{center} 
        \end{tcolorbox}
        \begin{tcolorbox}[enhanced, width=\linewidth, remember as=sco,top=1mm,bottom=1mm]
            \begin{center}Source Code Optimizer \end{center}
        \end{tcolorbox}
    \end{tcolorbox}
    \begin{tcolorbox}[enhanced jigsaw, breakable, noparskip,opacityframe=0.3, opacityback=0.3, height=0.7cm, width=\linewidth, remember as=me, watermark text=Middle-End]
    \end{tcolorbox}
    \begin{tcolorbox}[enhanced jigsaw, breakable, noparskip,opacityframe=0.3, opacityback=0.3, width=\linewidth, size=small]
        \begin{tcolorbox}[enhanced, width=\linewidth, remember as=cg,top=1mm,bottom=1mm]
            \begin{center}Code Generator  \end{center}
        \end{tcolorbox}
        \begin{tcolorbox}[enhanced, width=\linewidth, remember as=tco,top=1mm,bottom=1mm]
            \begin{center}Target Code Optimizer \end{center}
        \end{tcolorbox}
    \end{tcolorbox}
\end{tcolorbox}
\end{tcolorbox}
\begin{tikzpicture}[overlay,remember picture,line width=1mm]
    \draw[->, shorten >=-1.5mm] ($(cp.north)+(0,1)$) -- (sc.north);
    \draw[->, shorten >=-1.5mm] (sc.south) -- (ps.north);
    \draw[->, shorten >=-1.5mm] (ps.south) -- (sa.north);
    \draw[->, shorten >=-1.5mm] (sa.south) -- (sco.north);
    \draw[-] (sco.south) -- (me.north);
    \draw[->, shorten >=-1.5mm] (me.south) -- (cg.north);
    \draw[->, shorten >=-1.5mm] (cg.south) -- (tco.north);
    \draw[->] (tco.south) -- ($(cp.south)+(0,-1)$);
\end{tikzpicture}
\begin{tikzpicture}[overlay,remember picture,line width=1mm]
    \draw[-, draw=blue!30!white,line width=.5mm, shorten >=.2cm,shorten <=.1cm] (cmp.north east) -- (cp.north west);
    \draw[-, draw=blue!30!white,line width=.5mm, shorten >=.2cm,shorten <=.1cm] (cmp.south east) -- (cp.south west);
\end{tikzpicture}

        \caption{\label{fig:cmpintstruct} Schematic overview of different compiler stages.}
    \end{subfigure}
\end{figure}

What differs compilers from the competing concept of interpreters is the separation of \textbf{compile-time} and \textbf{run-time}.
As interpreters combine these two and translate a program at run-time, a compiler takes the time to read the source-language file completely (often several times) and only then creates the executable files which are executed after the process has finished.
The advantages of this are simple:
While a compiler takes some time at first until the program can be started, the resulting executable is next to always faster and more efficient.
This is due to the possibility of optimizing code during the compilation process and the chance of reading through the source file several times if this is needed (with each time the code is read being called a \textbf{pass}).
Of course there do exist many different compilers today and what matters to the user is typically the combination of the amount of time it takes to compile a program and the performance of that program.

A compiler is not the only contributor to translation of a program into an executable program although it is the most prominent one.
Figure~\ref{fig:compiler} illustrates the chain of tools that is involved into this process:
As one can see the \textbf{preprocessor} modifies the source before it is processed by the compiler and removes comments, substitutes macros and also includes other files into the source before it passes the new program code to the compiler.
The compiler then passes its output to the \textbf{assembler} which translates the output of the compiler which is written in a language called \textbf{assembly} into actual machine code by substituting the easy-to-read string alternatives with actual opcodes.
At last the \textbf{linker} combines the resulting ``object-files'' that the assembler emitted with standard library functions, that are already compiled, and other resources. 
The result is a single file that is almost executable.
The only task which is left for the \textbf{loader} is assigning a base address to the relative memory references of the ``relocatable'' file which were used until now.
The code is now fully written in machine language and ready for operation.
\\
\\
Since we are more interested in compilers than other components, we will take a better look at the compiler itself.
Figure~\ref{fig:compiler} shows the common separation of a compiler into \textbf{front-end}, \textbf{back-end} and an optional \textbf{middle-end}.
This is done to make a compiler portable, which means allowing the compiler to work for different source-languages which are implemented in the front-end and target-languages which must be specified in the back-end.
Therefore if one wants to compile two different programs e.g. one in C the other in FORTRAN, it is necessary to change the front-end but not the back-end because the machine or \textbf{target} stays the same.
The middle-end in this regard is not always needed but could be responsible for optimizations that are both source-independent and target-independent.

Of course the different parts of the compiler have to communicate through a language that all parts can understand or speak.
Such a language is called \textbf{intermediate representation} (IR) and also used during different phases of the compilation process.
It may differ in its form but always stays a representation of the original program code.
\\
\\
The different phases of a compilation process are illustrated to the far right of figure~\ref{fig:compiler}.
First the preprocessed source code is fed into the \textbf{scanner} that performs lexical analysis, which is combining sequences of characters to symbols and so called tokens that get associated with an attribute such as ``number'' or ``plus-sign'' and the symbol.
Next the \textbf{parser} takes the sequence of tokens and builds a syntax tree that represents the structure of a program and is extended by the \textbf{semantic analyzer} which adds known attributes at compile-time like ``integer'' or ``array of integers'' and checks if the resulting combinations of attributes are valid.
This already is the first form of IR.
The \textbf{source code optimizer} which is the last phase of the front-end takes the syntax tree and takes a first shot at optimizing the code.
Typically only light optimization is possible at this point such as pre-computing simple arithmetic instructions.
After the source code optimizer is done the syntax tree is converted into different IR in order to be passed to the back-end.

The \textbf{code generator} takes this IR and translates it to machine code that fits the target --- typically this is assembly.
At last the \textbf{target code optimizer} tries to apply target-specific optimization until the target code can be emitted.

During these phases the compiler also generates a symbol and literal table.
A \textbf{symbol table} is, as the name states, an overview of all symbols that are used in the program, it contains the symbols name and the attribute of the semantic analyzer.
A \textbf{literal table} in contrast holds constants (i.e. strings) and makes them globally available by reference, as does the symbol table.
This information is used by the code generator and various optimization processes.

\subsection{Back-End and Code Generation}
We now want to focus a little more on the last two phases of a compiler, which are also part of the back-end.
We already stated that the back-end is responsible for code generation and target optimization and since we will keep focus on the back-end later on, we need to get used to a few other terms that are common when talking about compiler back-ends.
\\
\\
Usually the processes of code generation and target optimization are entangled as optimization can take place at different phases of code generation.
Thus we first take a look at code generation in the back-end.

As we learned already, the source program reaches the back-end in form of IR.
Often the IR is already linearized and thereby again in a form that can be seen as sequence of instructions.
Because of this the IR may also be referred to as Intermediate Code.
The process of generating actual machine code from this is again split into different phases:
\begin{itemize}
    \item instruction selection
    \item instruction scheduling
    \item register allocation
\end{itemize}

At first the back-end recognizes sets of instructions in intermediate code that can be expressed as an equivalent machine instruction.
Depending on the complexity of the instruction set a single machine instruction can combine several IR instructions.
This may involve additional information that the front-end aggregated and added to the IR as attributes single machine instruction can combine several IR instructions.
At the end of this is a compiler typically emits a sequence of assembly instructions which we will explain later on.
In order to fulfill this task the compiler needs the specifications of the target it compiles for.
This is called a target description and can contain things like specifications of the register-set, restrictions and alignment in memory and availability of extensions and functionalities.
The compiler also needs knowledge of the instruction set of a target sometimes referred to as the ISA which is in essence a list of instructions which are available and also their functionality.
The compiler picks instructions according to their functionality from this list and substitutes the IR with this.
Ideally a back-end thus could support different back-ends just by exchanging the machine description and the ISA as the basic methods of generating code are the same for most targets.

After the IR is converted into machine instructions the back-end now rearranges the sequence of instruction.
This needs to be done as different instructions take different amounts of time to be executed.
If a subsequent instruction depends on the result of a previous instruction the compiler now has two alternative approaches to solve this.
First it can simply stall the programs execution as long as the instruction is executed and feed the next instruction into the processor only when the dependency is solved.
This means that the compiler adds |nop|s before every instruction that needs to wait for an operand as |nop| tells the processor to wait until the previous instruction has finished.
For critical memory usage the compiler can also insert |sync|s as memory barriers before hazardous memory instructions.
Alternatively it can stall only the instruction which depends on the result which is currently computed but perform instructions that do not depend on the result in the mean time.
By doing so the scheduler increases performance noticeably and thus can partly be seen as part of the optimization process.
On RISC architectures this is especially important as load and store instructions can take a few hundred times more clock cycles than normal register instructions and pipelining depends mainly on the instruction sequence.
Thus the scheduler is also involved parallelization of code.
As a result of this a compiler would usually accumulate all load instructions at he beginning of a procedure and start computing on registers that already have a value while the others are still loaded.
This is done vice versa at the end of a procedure for storing the results in memory.
This process of course needs the compiler to know the amount of time it takes for an instruction to be executed and works hand in hand with hazard detection on processor level.

At last he compiler handles register allocation which also includes memory handling.
Typically the previous processes expect an ideal target machine which provides and endless amount of registers.
As in reality the processor only has $k$ registers the register allocator reduces the number of ``virtual registers'' or ``pseudoregisters'' that are requested to the available number of ``hard registers'' $k$.
For this to be possible the compiler decides whether a value can live throughout a procedure in a register or must be placed into memory because there are not enough registers available.
This results in the allocator adding load and store instructions to the machine code in order to temporarily save those registers in memory which is called ``spilling''.
It is obvious that this can hurt performance and therefore the compiler tries to keep spilling of registers to a minimum and also insert spill code at places where it delays other instructions as little as possible.
At he end of register allocation the compiler assigns hard registers to the virtual registers which are now only $k$ at a time.

During and after code generation the compiler also applies optimizations to the machine code.
Any optimization to the code though must take three things into consideration, which are safety, profitability and risk/effort.
The first thing which always must be fulfilled, is safety.
Only if the compiler can guarantee that an optimization does not change the result of the transformed code compared to the original code it may use this optimization!
Only if this applies the compiler may check for the profit of an optimization which most often is a gain in performance but could also be the size of the program.
At last the effort or time it takes for the compiler to perform this optimization and the risk of generating actually bad or ineffective code should be taken into account as well.
If optimization passes these three aspects it may be applied to the code.
In the end there exist some simple optimizations that always pass this test like the deletion of unnecessary actions or unreachable code, e.g. functions that are never called.
Another example would be the reordering of code like the scheduler did before or the elimination of redundant code, which applies if the same value is computed at different points and thus the first result simply can be saved in a register.
If a compiler knows the specific costs of instructions, it can also try to substitute general instruction with more specialized but faster instructions, like substituting a multiplication with 2 by shifting a value one position to the left.
There exist many more ways of optimization but we only want to explain one more kind of optimization which is called peephole optimization.

In peephole optimization the compiler only looks at a small amount of code through a ``peephole'' and tries to find a substitution for this specific sequence of instructions.
These substitutions must be specified by hand and are highly target-dependent in contrast to the optimizations which were mentioned before that are target-independent.
If the sequence can be substituted the peephole optimizer does so, otherwise the peephole is moved one step further and the new sequence is evaluated.

\subsection{Assembly Basics}
\label{section:asm}
Following a short introduction in the previous section, we want to spend a little time on assembly as it is useful to know how to program in assembly while using C.
This is done as follows:
\begin{lstlisting}[caption= Exemplatory Assembly invkation, label=lst:asm]
asm volatile (  "add %0, %1, %2"
                : "=r" (dst)
                : "r" (src1), "r" (src2):);
\end{lstlisting}

The code above tells the compiler to generate the instruction |add| in assembly which is followed by three operands.
The number |n| in |%n| indicates that the operand is specified by the |n+1|th description of an operand that follows.
The description that follows after |:| describes the output operands.
|"=r"| means that the output is to be stored in a register (letter |r| for register operand) and that the register is to be written (|=| this is called constraint).
The variable in parentheses must be declared before its occurrence and of matching type (|float| would not be allowed in this case).
The following description is that of the input operands and those must not be written!
|r| again stands for a register operand and the variable is in parentheses, the arguments are separated by commas.
After the third |:| follow clobbered (=temporarily used) registers which would also be in quotes, but these are optional arguments.
|volatile| means that the compiler must not delete the following instructions due to optimization.

As a special command |asm (:::memory);| would indicate a memory barrier to the compiler, ergo the machine instructions previous to this line and the one following may not be interchanged.\unsure{keep this?}

\subsection{Intrinsics}
Something that will also occur quite often later in this thesis are intrinsics.
Intrinsics are sometimes also called built-in functions and resemble an intermediate form of |asm| and a high-level programming language.
This means that by calling an intrinsic function, we tell the compiler to use a certain machine instruction that typically shares its name with the intrinsic.
What differs an intrinsic from |asm()| is that we do not need to specify constraints or registers classes but only need to keep an eye on the type of arguments.
One could easily mistake them for normal functions of library but they are directly integrated into the back-end of a compiler and thus independent of the programming language.
In order to implement intrinsics into a back-end the compiler need a certain knowledge of what the |asm| instruction does and what kind of operands it needs.

A typical field of application for intrinsics would be vectorization and parallelization of code through processor extensions.
Sometimes this is the sole option of using the machine instructions associated with them.

\subsection{GNU Compiler Collection}
The GNU Compiler Collection (GCC) is a compiler suite that supports different programming languages and targets.
Though normally it is seen as a build of GCC supports a variety of front-ends while it was built for a specific target.
This target in most cases is the processor architecture on which the user runs the compiler.
But GCC also supports the idea of a cross-compiler which is the concept of compiling code on one machine but running the code on a different machine that is also based on a different architecture.
One must though build a version of GCC locally for every back-end one wants to compile code for.
This is realized through a modular structure which follows the idea of a front-end, middle-end and back-end as it was described in section \ref{section:compiler} although some information that belongs to a back-end is also needed at the front-end, hence the compiler is built back-end specific but supports a wide variety of back-ends to choose from.

GCC itself is programmed in C++ and part of the GNU project of the Free Software Foundation.
It is wide-spread and one of the most popular compilers especially among academic institutions and small scale developers.
Every major UNIX distribution and many minor ones include GCC as a standard compiler.\cite{definitveGCCGuide:introduction}
As an open source project there is a constant development to the compiler and there exist many threads that support known bugs.

There is one major competitor though which stands besides GCC as an open source compiler suite which is Clang that is part of the LLVM (low level virtual machine).
Both support running the same source code on multiple machines while LLVM actually runs intermediate code rather than actual machine code and uses GCC to generate this intermediate code for some front-ends.
However while one can argue in favor for either one, GCC seems a little better suited for our application. \todo{reference LLVM vs. GCC on ARM, LLVM vs. Gcc in EISC}
These results have to be viewed with care, as they are based on different processor architectures but it seems like both compilers provide similar performance.
Ultimately it is the personal preference of the programmer that decides which compiler one is more comfortable with and often enough he chooses that compiler.
In our case I have chosen GCC over LLVM for two main reasons.
One is that after all GCC follows more the traditional concept of a compiler that generates machine code at the end and also I was far more familiar with GCC than with LLVM when this decision had to be made.
The other is that GCC support existed to a minimum before I started this thesis and thus there was a point to start from.
This topic will be referred to later on in the discussion but for now a short motivation seems to be sufficient.

By now GCC is a stable release version of 6.3 with version 7 in the works but we will use an older version which is used internally at the BrainScaleS project which is version 4.9.2.
Additionally we will use binutils 2.25 which was patched by Simon Friedmann and since includes the opcodes and mnemonics which are supported by the nux.
A complete specification of the libraries used and a handy script that builds a cross-compiler for the nux on PowerPC systems can be found there as well.

We take a special interest in the PowerPC back-end of GCC which is called rs/6000 for IBMs RISC system/6000 architecture that is equivalent to POWER.
According to GCCs Internals manual \cite{GCCint}, which we will refer to as the sole source of information in this regard, the back-end of GCC has the following structure:

Each architecture has a directory with its respective name in gcc/config e.g. gcc/config/rs6000 that contains a minimum amount of files.
These are the machine description rs6000.md which is an overview of machine instructions with additional information to each instruction and the header files rs6000.h and rs6000-protos.h and source file rs6000.c that handle the target description through macros and functions.
Every back-end needs these files in the GCC source and the final back-end is build from these files through the macros and functions just mentioned.
To notice a back-end in the first place the back-ends directory -here "rs6000"- must be added to the file config.gcc which also includes a list of all files in the aforementioned directory.
Most back-ends include additional files which makes a back-ends complex structure clearer but these are not mandatory and we will address these later.

Instead we address one of the most important functions which unfortunately is also one of the least documented though most complicated ones.
The function/process is called ``reload'' and is used as part of the register allocation process. \cite{GCCwiki:reload}
Specifically reload is meant to do register spilling but over almost 25 years that GCC existed until 4.9.2 it became more and more complex and basically does everything associated with register allocation (mainly moving the contents of different registers and memory around, and finding the right registers in the first place).
Over the years it thus became one of the main sources of errors when constructing a back-end and was meant to be replaced several times.
As of now reload is being replaced by LRA (local register allocator) but GCC 4.9.2 is not impacted by this therefore we are stuck with reload indefinitely.

To address possible errors in reload later on we now get to know one form of IR in GCC that is Register Transfer Language (RTL).

\subsubsection{Register Transfer Language Basics}
RTL, which is not to be mixed up with Register Transfer Level, is a form of IR the back-end uses to generate machine code.
Usually GCC uses the IR GIMPLE which looks like stripped down C code with 3 argument expressions, temporary variables and |goto| control structures.
The back-end transforms this into a less readable IR that inherits GIMPLEs structure but brings it to a machine instruction level.
It is inspired by Lisp lists and thus we will need to take a look at those at last in before we take on the task of extending a GCC back-end.

We do so in looking at one of the most fundamental RTL statements first while explaining the each part at a time.

\begin{lstlisting}
(define_insn "add<mode>3"
  [(set (match_operand:VI2 0 "register_operand" "=v")
        (plus:VI2 (match_operand:VI2 1 "register_operand" "v")
		  (match_operand:VI2 2 "register_operand" "v")))]
  "<VI_unit>"
  "vaddu<VI_char>m %0,%1,%2"
  [(set_attr "type" "vecsimple")])

(define_insn "*altivec_addv4sf3"
  [(set (match_operand:V4SF 0 "register_operand" "=v")
        (plus:V4SF (match_operand:V4SF 1 "register_operand" "v")
		   (match_operand:V4SF 2 "register_operand" "v")))]
  "VECTOR_UNIT_ALTIVEC_P (V4SFmode)"
  "vaddfp %0,%1,%2"
  [(set_attr "type" "vecfloat")])
\end{lstlisting}

There exist manuals to basically everything which is written here and the more extensive manual will be referenced at he end of each paragraph.

|define_insn| is an RTL expression that generates an RTL equivalent to a machine instruction.
One such instruction is called an insn (short for instruction) hand has several properties like a name, an RTL template, a condition template, an output template and attributes. \cite{GCCint:defineinsn}
The name in this case is |add<mode>3| (|3| for three operands) where |<mode>| is to replaced by a set of values that describe a modes. \cite{GCCint:stdnames}
A mode is the form of an operand and can be something like |si| for single integer, |qi| for quarter integer (quarter the bits of a single integer), |sf| for single float or |v16qi| for a vector of 16 elements which are quarter integers each. \cite{GCCint:modes}
There are many more modes that follow the same scheme.
In this case we do not specify the mode explicitly but use an iterator that creates a |define_insn| for every valid mode we specify. \cite{GCCint:iterator}
The second |define_insn| shows this with a specific mode.

Next follows the RTL template which is in square brackets.
All RTL templates need a side effect expression as a base which describes what happens to the operands that follow.
In our case |set| means that the value which is specified by the second expression is stored into the place specified by the first expression. \cite{GCCint:sideeffect}
The first expression that follows is a specified operand.
|match_operand| tells the compiler that what follows is a new operand.
|VI2| belongs to the mode iterator we saw earlier and is to be substituted by the equivalent mode to <mode> in caps, which can be seen for the following |define_insn|.
All modes |VI2| are to be substituted by the same real mode.
After the mode comes the index of an operand which starts at 0 for every |define_insn|.
The following string describes a predicate which tells the compiler more about the operand and which constraints it must fulfill.
Operands typically end in |_operand| and a single predicate is meant to group several different operand types.
In this case any register would be a valid operand. \cite{GCCint:predicates}
The next string specifies the operands further and is meant to fine tune the predicate.
It is called a constraint and matches the description which was taken in section \ref{section:asm}.
|=| again means that the register must be writable and |v| stands for an AltiVec vector register.\cite{GCCint:constraints}
This pattern is repeated for every operand and only changes slightly.
Though the second expression of the |set| side effect has an additional pair of parentheses because of the |plus| statement.
This is an arithmetic expression and tells the compiler that the following operands are part of an operation that results in a new value.
It is also succeeded by a mode that specifies the mode of the result.\cite{GCCint:arith}

The RTL template is matched by the compiler against the RTL it generated from GIMPLE and if the template matches the RTL is substituted by the output template that follows.

After the RTL template is finished, the condition specifies if the insn may be used.
It is a C expression an must render |true| in order to allow the matching RTL pattern to be applied.
In this case the condition is also depending on the mode iterator which substitutes |<VI_unit>| for equivalent code to that of the next |define_insn| with a matching mode. \cite{GCCint:defineinsn}

The output template is usually is similar to the |asm| template from |asm|.
The string contains the mnemonic of a machine instruction and the operands which are numbered according to the indexes of the RTL template.
Again this is depending on the mode iterator and |<VI_char>| will be substituted by a character that belongs to a machine mode. \cite{GCCint:defineinsn}

At last the insn is completed by its attributes which hold further information about the insn that is used by the compiler internally like which effect an insn has on certain register etc..
We are less interested in this, as attributes are optional and we do not add attributes to the back-end. \cite{GCCint:attributes}


The attentive reader might have noticed that only RTL template is written in RTL.
This is true but still do insn patterns belong into this section.
The RTL not only is the most important part of an insn but we will hardly see RTL outside from RTL templates.
Still should RTL be mentioned in its other form here as it is used for debugging purposes.

RTL can be split into two phases which are non-strict RTL and strict RTL.
Non-strict occurs only before |reload| and is very deliberate in specifying its operands.
Operands usually are virtual registers that have a unique number.
|match_operand| then is replaced by |(reg:SI 1)| which tells the compiler the type of operand, the mode and the register number.\cite{GCCint:regsnmem}

Strict RTL has passed |reload| and no longer contains virtual registers but only references existing hard registers or memory.

An example of non-strict RTL and strict RTL of the same code can be seen in figure \add{examples of RTL code}

\add{
MMU ?= memory controller
PPC must handle syncing in compiler when I/O is added
explain stack and frame pointer
PPU instruction set
}
